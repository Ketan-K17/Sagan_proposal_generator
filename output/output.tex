\documentclass{article}
\usepackage{amsmath}
\usepackage{geometry}
\geometry{margin=1in}
\title{Advanced Asteroid Orbital Trajectory Determination}
\author{}
\date{}

\begin{document}

\maketitle

\section{Introduction: Originality of the Research Project}

\subsection{Context and Motivation}

The University of Luxembourg, through its Research Unit in Engineering Science (RUES), is committed to addressing the socio-economic needs and challenges of society and industry by becoming a leader in education and research in the Greater Region and globally. The unit focuses on three main research areas: Construction and Design, Energy and Environment, and Automation and Mechatronics. These areas encompass research into civil and mechanical engineering structures, energy efficiency, renewable energies, and dynamic testing methods, among others. The university aims to seamlessly integrate research and education to cultivate future leaders and critical thinkers.

In collaboration with over 70 private and public organizations through SnT’s Partnership Programme, the university addresses key challenges in ICT, contributing to the European Strategic Technology Plan and the Innovation Union in Europe. Since its launch in 2009, the Centre has rapidly developed, launching over 100 EU and ESA projects, protecting and licensing IP, and creating a dynamic interdisciplinary research environment with around 480 people.

For all AFR individual applications, a project idea must be outlined using a specific template, detailing the hypothesis, research questions, innovation, expected outcomes, and methodology. The FNR encourages the dissemination of research to the public and media, emphasizing the value and impact of research outputs. This approach ensures that research activities are aligned with industry, policymakers, and societal needs, fostering an innovation-driven research environment.

\subsection{Significance of the Project}

The FNR places significant emphasis on the impact of research outputs across science, industry, policy making, and society. To enhance this impact, FNR-funded research results are expected to be disseminated through high-quality, Open Access publications, in line with the FNR Policy on Open Access. The FNR also supports the deposition of preprints in open access repositories and offers an Open Access Fund to cover publication costs. Researchers are encouraged to list the value and impact of all research outputs, including preprints, publications, data, and intellectual property, and to plan for dissemination activities from the project's inception.

The SnT’s Partnership Programme exemplifies this approach by collaborating with over 70 private and public organizations to tackle key ICT challenges. Since its inception in 2009, SnT has rapidly developed, launching over 100 EU and ESA projects, protecting IP, and creating spin-offs. This dynamic environment aims to integrate research and education, focusing on energy, environment, and sustainable growth, aligning with the European Strategic Technology Plan and the Innovation Union.

LuxProvide complements these efforts by offering a platform that combines data science and supercomputing resources, aiding research and business players in Luxembourg and the Greater Region. Their approach emphasizes design thinking and co-creation, ensuring effective innovation. Public-private partnerships further enhance this ecosystem, allowing researchers to work closely with companies, spending a significant portion of their research period within the industry to achieve innovation and optimization goals.

\section{Hypothesis, Research Objectives, and Envisaged Methodology}

\subsection{Hypothesis and Research Questions}

Applicants are expected to demonstrate the value and impact of their research outputs, which include preprints, research publications, data, reagents, software, intellectual property, and the training of young scientists. The FNR encourages the dissemination of research to the general public and media, emphasizing the need for impact-generating activities from the project's inception. Evaluations should focus on the content and quality of scientific outputs rather than their quantity or the venue of publication, in line with the DORA Declaration. This approach values diverse research-related and non-research-related outputs, recognizing that important contributions vary across disciplines and individuals. Statistical analyses, such as the Friedman test and subsequent Nemenyi post-hoc test, are used to assess the significance of research findings, ensuring that proposed methods are statistically validated. The evaluation process should consider only the information submitted by the applicant, including the proposal, attachments, and ORCID profile, to ensure a comprehensive assessment of the research's quality and impact.

\subsection{Research Objectives}

Applicants for research funding are expected to demonstrate the value and impact of their research outputs, including preprints, publications, data, software, and intellectual property, while also training young scientists. The FNR, a signatory of the DORA declaration, emphasizes the importance of focusing on the scientific content of research rather than relying on journal-based metrics like Journal Impact Factors. Dissemination of research to the public and media is encouraged, and activities aimed at generating impact should be planned from the project's inception.

For AFR individual applications, a project idea must be outlined using a specific template, detailing the hypothesis, research questions, innovation, expected outcomes, and methodology. This document should be converted to PDF and uploaded as part of the application process.

In the context of industry partnerships, the focus is on quality, price, and deadlines. Companies use services to validate concepts, detect design problems, and test solutions to quickly introduce new products to the market. This aligns with the research areas of construction and design, energy and environment, and automation and mechatronics, which aim to integrate research and education to develop future leaders and critical thinkers.

The research also involves space mission design, where project life cycles are compared between ESA and NASA. The mission's scientific goals, performance, and technical requirements are established in the early phases, with the potential for modifications based on new findings. Knowledge management practices are crucial for capturing and sharing knowledge to improve organizational efficiency, as demonstrated by the DEA project, which restructures data from various sources to enhance processes.

\subsection{Methodology}

Our organization is dedicated to seamlessly integrating research and education to cultivate future leaders and critical thinkers. Our research activities are organized into three main areas: Construction and Design, focusing on civil and mechanical engineering structures; Energy and Environment, emphasizing energy efficiency and renewable energies; and Automation and Mechatronics. We also specialize in cybersecurity, offering services such as risk assessment and incident management to help businesses achieve their goals. Our consultancy services support hi-tech and telecommunication projects at all stages, from initial studies to project completion, providing critical business decision-making insights through targeted analytics. We conduct feasibility studies that produce outputs like study reports and Engineering Models, which are crucial for project planning and implementation. Our holistic suite of services aids companies in new product and company development, from ideation to commercialization. Our organization, a small and medium-sized enterprise founded in 1993, is led by CEO Jean-Paul Henry and is recognized as an ESA BASS BROKER. We continuously evaluate our findings and contributions, addressing current limitations and recommending future work to enhance our methodologies and support tools.

\section{Expected Outcomes / Impact}

\subsection{Scientific and Societal Impact}

The FNR places significant emphasis on the impact of research outputs on science, industry, policy making, and society at large. Applicants are expected to detail the value and impact of their research outputs, including preprints, publications, data, and other intellectual contributions. The FNR encourages the dissemination of research through high-quality, Open Access publications, in line with its Open Access Policy, and supports this through its Open Access Fund. Additionally, the FNR promotes the deposition of preprints in open access repositories to maximize research impact. Evaluation of proposals should focus on the content and quality of scientific outputs, rather than journal-based metrics, in accordance with the DORA Declaration. Reviewers are advised to consider only the information submitted by the applicant, including the proposal and ORCID profile, and to assess the proposal's objectives and fulfillment of selection criteria. Panel members must declare conflicts of interest before discussing proposals. The FNR also encourages activities aimed at generating impact from the initial project planning stage, ensuring that research benefits are communicated to the general public and media.

\subsection{Dissemination and Collaboration}

The FNR places significant emphasis on the impact of research outputs across science, industry, policy making, and society. Applicants are expected to detail the value and impact of their research outputs, which can include preprints, publications, data, reagents, software, intellectual property, and the training of young scientists. The FNR encourages the dissemination of research to the general public and media, and activities aimed at generating impact should be planned from the project's inception.

In evaluating proposals, reviewers are advised to focus on the content and quality of scientific outputs rather than their quantity, publication venue, or journal metrics. The FNR, a signatory of the DORA Declaration, values all types of research outputs, recognizing that important outputs vary across disciplines and individuals.

To maximize research impact, FNR-funded research results are expected to be disseminated through high-quality, Open Access publications, in line with the FNR Policy on Open Access. Costs for project-related publications can be refunded through the FNR’s Open Access Fund. The FNR also supports the deposition of preprints in open access repositories.

The SnT’s Partnership Programme exemplifies this approach by collaborating with over 70 organizations to address key challenges in ICT, fostering a dynamic interdisciplinary research environment. This initiative has led to significant developments, including the recruitment of top scientists, the launch of numerous projects, and the creation of spin-offs, all contributing to the broader impact of research on industry and society.

\section{Explanations on the Management of Ethical Issues and Data Protection}

\subsection{Ethical Compliance}

The FNR-funded research activities, both inside and outside academia, must adhere to several general principles to ensure ethical and impactful research. All research activities should respect fundamental ethical principles, as outlined in the Charter of Fundamental Rights of the European Union, and comply with the FNR Research Integrity Guidelines. Any research misconduct, such as non-compliance with ethical regulations, provision of false information, plagiarism, or data falsification, may lead to proposal rejection and further actions by the FNR.

Host institutions are responsible for obtaining all necessary authorizations from ethical and data protection committees or other regulatory bodies. Any ethical misconduct can result in the immediate suspension or termination of the grant, with potential additional sanctions, including reimbursement requests and legal actions.

The FNR emphasizes the importance of research impact on science, industry, policy makers, and society. Applicants are expected to demonstrate the value and impact of their research outputs, including preprints, publications, data, software, intellectual property, and training of young scientists. Dissemination of research to the public and media is encouraged, and activities aimed at generating impact should be planned from the project's inception.

Beneficiaries must comply with the FNR research integrity guidelines and ethical charter, available on the FNR website. The FNR endorses the European Code of Conduct for Research Integrity and the Singapore Statement on Research Integrity. The merit review process follows international standards of transparency, impartiality, confidentiality, and integrity, as defined by the Global Summit on Merit Review.

Panel experts are required to read and abide by the FNR Ethics Charter and Code of Conduct for Research Assessment, observing principles of transparency and integrity. They should also consider legitimate delays in research activity due to personal factors, such as parental leave or disability.

All electronic materials and publications should prominently display the FNR logo and acknowledge FNR funding. This comprehensive approach ensures that FNR-funded research is conducted with the highest ethical standards and contributes significantly to scientific and societal advancement.

\subsection{Data Protection and Management}

The FNR (Fonds National de la Recherche) has established a comprehensive framework for the evaluation and management of research proposals, ensuring confidentiality, ethical compliance, and transparency throughout the process. Panel members and experts involved in the evaluation must declare their commitment to confidentiality and are prohibited from using the data for personal purposes. They are required to read relevant documents, sign a participation form, and thoroughly review assigned proposals. Conflicts of interest must be declared, and any inability to fulfill obligations must be reported to the FNR immediately.

Applicants and Host Institutions (HI) agree to the publication of certain information, such as the beneficiary's name and nationality, and consent to the sharing of their full application with expert panel members for evaluation purposes. The project abstract, used to contact external experts, must not contain confidential information.

In case a beneficiary cannot pursue their project due to unforeseen circumstances, they must inform the FNR within 30 days. The FNR will communicate decisions regarding requests within a reasonable timeframe. Any ethical misconduct may lead to suspension or termination of the grant, with possible additional sanctions.

The FNR emphasizes the importance of research impact on various sectors and ensures compliance with GDPR, allowing beneficiaries to access and modify their personal data. The FNR also reserves the right to monitor ongoing grants and may introduce measures like anonymized surveys to ensure proper documentation and development plans for beneficiaries.

\section{Comment on Resubmission (if applicable)}

\subsection{Evaluation Process}

The evaluation process for research proposals involves several key steps to ensure a thorough and fair assessment. Initially, panel experts are required to complete and submit their reviews, providing comments and scoring the proposals at least 10 working days before the panel meeting. During the meeting, these experts orally present their reviews, focusing on the proposal's objectives, strengths and weaknesses, and the fulfilment of selection criteria. They also highlight any conflicting statements from reviewers and provide an overall assessment, including necessary modifications if applicable.

The panel meeting, organized by the FNR, begins with a ranking of proposals based on the reviews received. Proposals that do not meet the minimal quality criteria are not discussed further, although the justification for their low scoring is verified. Panel members are reminded to declare any conflicts of interest before proceeding with the discussion of proposals. The panel chair invites members to discuss the evaluation findings and adjust the proposal ratings if necessary, providing argumentation for any changes.

After rating all proposals, the panel discusses which should benefit from funding programs. The applicant and the Host Institution (HI) must agree to the Terms and Conditions, which form the contractual basis of the grant agreement with the FNR. They also acknowledge the FNR's commitment to the principles of the European Charter for Researchers.

Applicants are expected to demonstrate the value and impact of their research outputs, including publications, data, and intellectual property. The FNR encourages the dissemination of research to the public and media, emphasizing the need for activities aimed at generating impact from the project's inception.

The panel meeting concludes with the preparation of a 'Panel Conclusion' for each project, summarizing the discussions and decisions made. This comprehensive evaluation process ensures that only high-quality proposals aligned with the FNR's objectives and societal impact goals are selected for funding.

\subsection{Resubmission Guidelines}

The evaluation process for funding proposals involves a structured approach where each proposal is reviewed by two experts who are closest to the domain of the proposal. These experts provide a synthesis of the written evaluations, focusing on the proposal's objectives, fulfillment of selection criteria, and any conflicting statements from reviewers. The panel, which is nominated annually by the FNR, consists of generalists who may not be experts in the specific domain but are responsible for discussing each proposal in a panel meeting.

During the panel meeting, the panel expert presents the strengths and weaknesses of the proposal, along with an overall assessment and any necessary modifications. Ethical considerations are also addressed if applicable. The panel members engage in discussions, raise concerns, and ask questions, regardless of whether the proposal is in their field. They are reminded to declare any conflicts of interest.

After discussing the proposals, the panel members rate them, and if necessary, readjust the ratings based on the discussion. A consensus on funding recommendations is sought, and if a two-thirds majority is reached, a positive funding recommendation is made. The 'Panel Conclusion' is prepared for each project and is the only feedback sent to the applicant. The final funding decision is communicated to the applicant and their supervisor(s).

Additionally, the process emphasizes the importance of a feedback loop, allowing for continuous improvement and learning. This could involve user comments on query outputs, as seen in tools like Wolfram Alpha and Google, ensuring the tool remains user-friendly and adaptable.

\section{Bibliography (max. 15 references, not included in character limits)}

\subsection{Sources and References}

The text collection is built on four heterogeneous data sources: ESA feasibility reports generated during CDF studies, academic publications, books, and Wikipedia pages. These sources are crucial for research and knowledge dissemination, as highlighted by experts during the 2018 survey. The ESA CDF reports, along with Engineering Models, are primary outputs of CE studies, summarizing mission objectives and requirements. Academic publications and books are peer-reviewed, ensuring content is verified by humans, while Wikipedia serves as a common open-source knowledge base. The FNR emphasizes the importance of research impact on science, industry, policy making, and society. To maximize this impact, FNR-funded research results are expected to be disseminated via high-quality, Open Access publications, with costs potentially refunded through the FNR’s Open Access Fund. The SnT’s Partnership Programme collaborates with over 70 organizations, addressing key challenges in ICT and fostering a dynamic interdisciplinary research environment. The University of Strathclyde KnowledgeBase and other platforms like Science and NASA's knowledge graph are utilized for accessing recent research findings and critical data.

\end{document}