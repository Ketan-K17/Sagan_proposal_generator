```latex
\documentclass[12pt]{article}
\usepackage{geometry}
\geometry{a4paper, margin=1in}
\usepackage{amsmath}
\usepackage{hyperref}

\title{Advanced Rocket Engines Development: A Comprehensive Research Project}
\author{}
\date{}

\begin{document}

\maketitle

\section{Introduction: Originality of the Research Project}

\subsection{Integration of Research, Industry, and Policy-Making}

The integration of research, industry, and policy-making is pivotal in addressing societal challenges, as emphasized by the Fonds National de la Recherche (FNR). The FNR underscores the importance of research outputs, which include preprints, publications, data, software, and the training of young scientists. Dissemination to the public and media is encouraged, necessitating impact-generating activities from the project's inception. The SnT Partnership Programme exemplifies collaboration with over 70 organizations, tackling ICT challenges and fostering innovation through EU and ESA projects, IP protection, and spin-offs. This dynamic environment, with 480 members, supports interdisciplinary research and real-world application of prototypes.

In space mission design, the trend towards complex systems underscores the importance of knowledge reuse, particularly in early stages. The ESA CDF's study portal and metadata tools facilitate heritage analysis, although human expertise remains vital. Concurrent Engineering (CE) accelerates feasibility studies, with knowledge management and reuse being pivotal. The survey identifies a knowledge bottleneck in early design stages, emphasizing the need for heritage information and experienced colleagues. This comprehensive approach ensures that research and design processes are informed, innovative, and impactful.

The FNR places significant emphasis on the impact of research outputs on science, industry, policy making, and society. To maximize this impact, FNR-funded research results are expected to be disseminated through high-quality, Open Access publications, in line with the FNR Policy on Open Access. The FNR also supports the deposition of preprints in open access repositories and offers an Open Access Fund to cover publication costs. Applicants are encouraged to list the value and impact of all research outputs, including preprints, publications, data, and intellectual property, and to plan activities aimed at generating impact from the project's inception.

The FNR aims to become a leader in education and research in the Greater Region and a global player in core research areas, with a focus on energy, environment, and sustainable growth. This aligns with the European Strategic Technology Plan and the EU's Innovation Union goals. The FNR fosters an innovation-driven research environment that integrates research and education to train future scientists.

Public-private partnerships (PPP) are a key component of FNR's strategy, with research projects jointly developed by applicants, companies, and public partners. Researchers spend a significant portion of their research period in companies, enhancing collaboration and real-world application of research.

Through the SnT Partnership Programme, researchers collaborate with over 70 private and public organizations to tackle key challenges in ICT. Since its launch in 2009, SnT has rapidly developed, recruiting top scientists, launching numerous EU and ESA projects, protecting IP, and creating spin-offs. This dynamic interdisciplinary environment supports innovation and the development of prototypes in real-world settings.

Additionally, initiatives like ESA BASS and Creaction support Luxembourg-based start-ups and SMEs in integrating space applications to meet innovation needs, offering feasibility studies and demonstration projects. This holistic approach ensures that research outputs are impactful and aligned with industry and societal needs.

\subsection{Trends in Space Mission Design}

The FNR encourages applicants to highlight the value and impact of their research outputs, including preprints, publications, data, software, intellectual property, and the training of young scientists. Dissemination of research to the public and media is also encouraged, with impact-generating activities planned from the project's inception. Applicants should use the provided template to outline their project idea, hypothesis, innovation, expected outcomes, and methodology, ensuring the document is converted to PDF for submission. The FNR, a DORA Declaration signatory, advises against using journal-based metrics like Journal Impact Factors to assess research quality, focusing instead on scientific content. The SnT Partnership Programme collaborates with over 70 organizations, addressing ICT challenges and fostering a dynamic research environment. This includes major space projects and initiatives like the ESA Incubed project "FloodSENS" and participation in NASA/Europe Frontiers Development Lab. The programme supports innovation through partnerships, providing researchers with real-world data and systems for testing, often resulting in prototypes for partners. The Ministry of Foreign and European Affairs, through projects like MILAN and ECOSTRESS, focuses on space-related research, while expertise in digital platforms and data valorization supports ongoing projects like Digital Twins and Secured Digital Platforms for Earth Observation.

The FNR places significant emphasis on the impact of research outputs on science, industry, policy making, and society. To maximize this impact, FNR-funded research results are expected to be disseminated through high-quality, Open Access publications, in line with the FNR Policy on Open Access. This includes the deposition of preprints in open access repositories, with costs for project-related publications refundable through the FNR’s Open Access Fund. Applicants are encouraged to list the value and impact of all research outputs, including preprints, publications, data, and intellectual property, and to plan activities aimed at generating impact from the project's inception.

Research activities are organized into three main areas: Construction and Design, Energy and Environment, and Automation and Mechatronics. These areas focus on civil and mechanical engineering, energy efficiency, and renewable energies, respectively. The integration of research and education is crucial to forming future leaders and critical thinkers.

The SnT’s Partnership Programme exemplifies collaboration with over 70 private and public organizations, addressing key challenges in ICT. Since its launch in 2009, the Centre has rapidly developed, recruiting top scientists, launching over 100 EU and ESA projects, and creating a dynamic interdisciplinary research environment with around 480 people. This environment fosters the protection and licensing of IP, the launch of spin-offs, and the seamless integration of research and education.

Furthermore, the importance of heritage knowledge in research is highlighted, as it accelerates the study phase by providing reliable and synthesized information. This is particularly relevant in fields like space mission design, where accumulated data from past missions can inform current projects. The FNR encourages the dissemination of research to the general public and media, ensuring that the impact of research extends beyond academia to benefit society as a whole.

\subsection{Impact and Dissemination of Research}

The integration of research, education, and innovation is crucial for addressing the needs of industry, policymakers, and society at large. Applicants are encouraged to highlight the value and impact of their research outputs, including publications, data, software, and training of young scientists. The FNR emphasizes the importance of disseminating research to the public and media, suggesting that impact-generating activities should be planned from the project's inception. Our research is organized into three main areas: Construction and Design, Energy and Environment, and Automation and Mechatronics, with a focus on energy efficiency, renewable energies, and sustainable growth. This aligns with the European Strategic Technology Plan and the Innovation Union initiative, aiming to position the Greater Region as a leader in education and research. The University of Strathclyde, under the guidance of Prof. Massimiliano Vasile, exemplifies this integration by fostering concurrent design studies and collaborative efforts, such as the historic 1956 Dartmouth conference that laid the foundation for AI research. LuxProvide further supports this ecosystem by offering a platform that combines data science and supercomputing resources, emphasizing a design thinking and co-creation approach to drive innovation.

\section{Hypothesis, Research Objectives, and Envisaged Methodology}

\subsection{Evaluation and Selection Process}

The evaluation and selection process for research proposals, such as those for the AFR 2024 Call, emphasizes the quality and impact of scientific outputs over traditional journal-based metrics. This approach aligns with the principles of the Declaration on Research Assessment (DORA), which the FNR has adopted. Reviewers are encouraged to consider a wide range of research outputs, including preprints, publications, data, software, and intellectual property, as well as the training of young scientists. The process involves an administrative eligibility check followed by a detailed review focusing on the content and quality of the submissions. Applicants are expected to demonstrate the value and impact of their research on industry, policymakers, and society, with an emphasis on dissemination to the public and media. The methodology and approach sections of the proposals are crucial, as they outline the strategies for achieving the desired outcomes, reducing costs, and mitigating risks. The evaluation also considers the potential for future research directions and the overall contributions of the findings.

The FNR, as a signatory of the DORA Declaration, emphasizes the importance of evaluating research quality and impact independently of journal-based metrics. Applicants are encouraged to list a diverse range of research outputs, including datasets, software, intellectual property, and the training of young scientists, rather than relying on journal impact factors. The focus should be on the scientific content and quality of outputs, recognizing that important contributions can vary across disciplines and may extend beyond traditional research articles. The FNR also supports the dissemination of research to the general public and media, and encourages activities aimed at generating impact from the initial project planning stage. In evaluating proposals, reviewers should consider only the information submitted by the applicant, including the proposal, attachments, and ORCID profile, and give value to all types of research outputs. Additionally, fostering a positive research culture with an emphasis on diversity and inclusion is encouraged, aligning with the European Commission’s recommended practices.

Our research and education initiatives aim to seamlessly integrate efforts to form future leaders and critical thinkers, focusing on three main areas: Construction and Design, Energy and Environment, and Automation and Mechatronics. We collaborate with over 70 private and public organizations through SnT’s Partnership Programme, addressing key challenges in ICT and fostering innovation. Our research outputs, including preprints, publications, data, and software, are expected to demonstrate value and impact on industry, policy makers, and society. The FNR encourages dissemination of research to the public and media, emphasizing ethical principles and research integrity. We prioritize content and quality of scientific outputs over quantity, recognizing diverse outputs beyond research articles. Our research culture promotes diversity, inclusion, and gender equity, aligning with the European Charter for Researchers. FNR-funded PhDs are provided a supportive research and training environment, adhering to the National Quality Framework for Doctoral Education.

\subsection{Methodology and Approach}

The FNR, as a signatory of the Declaration on Research Assessment (DORA), has reformed its evaluation process for research proposals to focus on the quality and impact of scientific content rather than relying on journal-based metrics like Journal Impact Factors. Applicants are encouraged to list a diverse range of research outputs, including preprints, datasets, software, intellectual property, and the training of young scientists, while emphasizing the scientific content over publication metrics. The FNR also promotes the dissemination of research to the general public and media, and activities aimed at generating impact should be integrated from the initial project planning stages. The AFR review process involves an administrative eligibility check and emphasizes fostering a positive research culture by encouraging diversity and inclusion, such as developing gender equity plans. Additionally, consultancy services support hi-tech and telecommunication projects through all stages, from initial studies to project completion, ensuring critical business decision-making is informed by targeted analytics and metadata.

The FNR-funded research activities are designed to create significant value and impact across industry, policy makers, and society at large. Applicants are expected to detail the value and impact of their research outputs, including preprints, publications, data, reagents, software, intellectual property, and the training of young scientists. The FNR also emphasizes the importance of disseminating research to the general public and media, encouraging activities that generate impact from the project's inception. Research activities must adhere to fundamental ethical principles, as outlined in the Charter of Fundamental Rights of the European Union, and comply with the FNR Research Integrity Guidelines. Misconduct, such as non-compliance with ethical regulations, provision of false information, plagiarism, or data falsification, may lead to proposal rejection and further actions by the FNR.

The SnT Partnership Programme exemplifies this approach by collaborating with over 70 private and public organizations to tackle key challenges in ICT. Since its inception in 2009, the Centre has rapidly developed, recruiting top scientists, launching over 100 EU and ESA projects, protecting and licensing IP, and creating a dynamic interdisciplinary research environment. The iterative design process, as illustrated by the Spiral Model, is critical in feasibility studies, which produce outputs like study reports and Engineering Models (EMs). These studies, often conducted in collaboration with experienced colleagues, utilize metadata and analytics algorithms to support decision-making and optimize operations. The ESA CDF and CCDS exemplify this process through structured sessions and student-led design challenges, ensuring that research outputs have a lasting impact on science and society.

\subsection{Collaboration and Innovation}

The FNR places significant emphasis on the impact of research outputs on science, industry, policy making, and society at large. Applicants are expected to detail the value and impact of their research outputs, including preprints, publications, data, and other intellectual contributions. The FNR encourages the dissemination of research through high-quality, Open Access publications, in line with its Open Access Policy, and supports this through its Open Access Fund. Additionally, the FNR promotes the deposition of preprints in open access repositories to maximize research impact. Evaluation of proposals should focus on the content and quality of scientific outputs, rather than journal-based metrics, in accordance with the DORA Declaration. Reviewers are advised to consider only the information submitted by the applicant, including the proposal and ORCID profile, and to assess the proposal's objectives and fulfillment of selection criteria. Panel members must declare conflicts of interest before discussing proposals. The FNR also encourages activities aimed at generating impact from the initial project planning stage, ensuring that research benefits are communicated to the general public and media.

The FNR places significant emphasis on the impact of research outputs on science, industry, policy making, and society at large. Applicants are expected to demonstrate the value and impact of their research outputs, which can include preprints, research publications, data, reagents, software, intellectual property, and the training of young scientists. The FNR encourages the dissemination of research to the general public and media, and activities aimed at generating impact should be planned from the project's inception.

The focus should be on the content and quality of scientific outputs rather than their quantity, publication venue, or journal metrics. It is important to consider a diverse range of research-related and non-research-related outputs, which may vary across disciplines and individuals. These outputs can extend beyond research articles to include data, reagents, software, mentoring, societal outreach, and policy changes.

The FNR mandates that research results be disseminated through high-quality, Open Access publications in line with its Open Access Policy. Costs for project-related publications can be refunded through the FNR’s Open Access Fund. Additionally, the FNR supports the protection and economic exploitation of research results, expecting an appropriate IP protection and exploitation strategy at the host institution.

Through initiatives like SnT’s Partnership Programme, researchers collaborate with over 70 private and public organizations to tackle key challenges in ICT, contributing to a dynamic interdisciplinary research environment. The FNR also acknowledges legitimate delays in research activities due to personal factors and encourages the deposition of preprints in open access repositories to maximize research impact.

\section{Expected Outcomes / Impact}

\subsection{Impact on Science, Industry, and Society}

The FNR places significant emphasis on the impact of research outputs across science, industry, policy-making, and society. Applicants are encouraged to list a diverse range of research outputs, including preprints, publications, data, reagents, software, intellectual property, and the training of young scientists. The FNR advocates for the dissemination of research to the general public and media, and activities aimed at generating impact should be integrated from the initial project planning stage. As a signatory of the DORA declaration, the FNR advises against using journal-based metrics like Journal Impact Factors as a measure of quality, urging a focus on the scientific content of outputs instead. The FNR also recognizes legitimate delays in research activity due to personal factors and encourages sensitivity towards these circumstances. To maximize the impact of research, FNR-funded projects are expected to be disseminated through high-quality, Open Access publications, with costs potentially covered by the FNR’s Open Access Fund. Compliance with FNR Research Integrity Guidelines is mandatory, and any research misconduct may lead to proposal rejection and further actions.

The FNR emphasizes the importance of focusing on the content and quality of scientific outputs rather than their quantity, publication venue, or journal-based metrics like Journal Impact Factors. As a signatory of the DORA Declaration, the FNR encourages applicants to list a diverse range of research outputs, including preprints, research publications, data, reagents, software, intellectual property, and the training of young scientists. These outputs should be evaluated based on their value and impact on science, industry, policy-making, and society. The FNR also promotes the dissemination of research to the general public and media, advocating for Open Access publications in line with its policy. Reviewers are advised to assess proposals based on the applicant's submitted information, including their ORCID profile, and to value all types of research outputs independently of journal-based metrics. This approach ensures that the impact of research is maximized and aligned with the FNR's commitment to high-quality, accessible scientific contributions.

The FNR places significant emphasis on the impact of research outputs on science, industry, policy making, and society at large. Applicants are expected to demonstrate the value and impact of their research outputs, which extend beyond traditional research articles to include data, reagents, software, intellectual property, and the training of young scientists. The FNR encourages the dissemination of research to the general public and media, and mandates that results from FNR-funded research be published in high-quality, Open Access publications, with costs potentially covered by the FNR’s Open Access Fund. The focus should be on the content and quality of scientific outputs rather than their quantity or the prestige of the publication venue. The FNR also recognizes the importance of considering legitimate delays in research activities due to personal factors such as parental leave or disability. Additionally, the FNR supports the introduction of measures to reduce the influence of less reliable sources and encourages a broad range of research-related outputs, acknowledging that important outputs can vary across disciplines and individuals.

\subsection{Dissemination and Exploitation of Research}

The FNR-funded research activities, both inside and outside academia, must adhere to several general principles to ensure integrity and impact. All research activities should respect fundamental ethical principles, as outlined in the Charter of Fundamental Rights of the European Union, and comply with the FNR Research Integrity Guidelines. Any research misconduct, such as non-compliance with ethical regulations, provision of false information, plagiarism, or data falsification, may lead to proposal rejection and further actions by the FNR.

Host institutions are responsible for obtaining all necessary authorizations from ethical and data protection committees or other regulatory bodies. Any ethical misconduct can result in the immediate suspension or termination of the grant, with potential additional sanctions, including reimbursement requests and legal actions.

The FNR emphasizes the importance of research outputs' impact on science, industry, policy makers, and society. Applicants should list the value and impact of all research outputs, including preprints, publications, data, reagents, software, intellectual property, and training of young scientists. Dissemination of research to the public and media is encouraged, and activities aimed at generating impact should be planned from the project's inception.

The FNR is committed to a merit review process that upholds international standards of transparency, impartiality, confidentiality, and integrity, as defined by the 2012 Global Summit on Merit Review. Panel experts must read and adhere to the FNR Ethics Charter and Code of Conduct for Research Assessment, observing principles of transparency and integrity.

Beneficiaries must comply with the FNR research integrity guidelines and ethical charter, available on the FNR website. The FNR endorses the European Code of Conduct for Research Integrity and the Singapore Statement on Research Integrity. All electronic materials and publications should prominently display the FNR logo and acknowledge FNR funding.

Research outputs should extend beyond articles to include data, reagents, software, mentoring, societal outreach, intellectual property, and policy changes. The FNR is sensitive to legitimate delays in research activity due to personal factors, such as parental leave, part-time work, or disability, which may affect an applicant's research record.

\subsection{Alignment with FNR and EU Goals}

The FNR-funded research activities, both inside and outside academia, must adhere to several key principles to ensure integrity and ethical compliance. All research proposals submitted under the CORE programme are required to respect fundamental ethical principles, as outlined in the Charter of Fundamental Rights of the European Union, and comply with the FNR Research Integrity Guidelines. Any form of research misconduct, such as non-compliance with ethical regulations, provision of false information, plagiarism, or data falsification, can lead to proposal rejection and further actions by the FNR.

The FNR emphasizes the importance of the impact of research outputs on science, industry, policy makers, and society. Applicants are expected to demonstrate the value and impact of their research outputs, including preprints, publications, data, and intellectual property. Dissemination of research to the public and media is encouraged, and should be planned from the project's inception.

Beneficiaries must ensure compliance with the FNR ethical charter and research integrity guidelines, which are available on the FNR website. The FNR endorses the European Code of Conduct for Research Integrity and the Singapore Statement on Research Integrity. All projects must also comply with the EU’s General Data Protection Regulation (GDPR) regarding data protection issues, and obtain necessary authorizations from relevant ethical and regulatory bodies.

The FNR is committed to a merit review process that is transparent, impartial, confidential, and adheres to international standards of integrity. Panel experts are required to read and abide by the FNR Ethics Charter and Code of Conduct for Research Assessment, and to sign a Participation Form before reviewing proposals.

Data management and open access are also critical components, with researchers expected to follow good practices in data handling, management, protection, and security, in line with the FNR Policy on Research Data Management and the FAIR principles.

Finally, all electronic materials and publications should prominently display the FNR logo and acknowledge FNR funding, ensuring proper recognition of the support provided by the Fonds National de la Recherche, Luxembourg.

\section{Explanations on the Management of Ethical Issues and Data Protection}

\subsection{Adherence to Ethical Principles}

The text outlines a comprehensive approach to data protection, cybersecurity, and infrastructure management, emphasizing compliance with the EU's General Data Protection Regulation (GDPR). It highlights the importance of protecting critical infrastructure, space ICT, and computer forensics, offering a range of products and services such as consulting, risk analysis, and forensic analysis. The text also underscores the significance of data management and protection in research, with a focus on ensuring data integrity and security through innovative solutions. Organizations are encouraged to establish data management plans, ensure GDPR compliance, and utilize state-of-the-art digital solutions for data traceability and protection, particularly in space, government, and defense sectors. The text emphasizes the need for appropriate stewardship of research data, adherence to FAIR principles, and the importance of data privacy impact assessments. Additionally, it highlights the role of cybersecurity companies like FACTiven in safeguarding space system data, ensuring its reliability and security to support informed decision-making in various fields, including climate change adaptation and disaster relief.

\subsection{Data Protection and Management}

Before the panel meeting, experts are required to carefully read the Programme Description, the FNR Ethics Charter, the Code of Conduct for Research Assessment, and the Peer Review Guidelines. They must sign a 'Participation Form' and thoroughly read the assigned proposals. Panel members are expected to maintain confidentiality, impartiality, and declare any conflicts of interest, withdrawing from tasks if necessary. Research activities funded by the CORE programme must adhere to fundamental ethical principles, including those in the Charter of Fundamental Rights of the European Union, and comply with the FNR Research Integrity Guidelines. Misconduct, such as non-compliance with ethical regulations, provision of false information, plagiarism, or data falsification, may lead to proposal rejection and further actions by the FNR. The FNR emphasizes the impact of research outputs on science, industry, policy-making, and society. Beneficiaries have rights under the GDPR to access and modify their personal data, and the FNR ensures data protection compliance. The applicant and Host Institution must secure necessary authorizations from ethical and data protection committees. Misconduct may result in grant suspension, termination, or additional sanctions, including reimbursement requests and legal actions. A Consortium agreement detailing partner contributions, IPR, and collaboration, including training and career development, must be signed at the project start. The publication of research results, including PhD theses, should be allowed, with restrictions imposed only when necessary.

\subsection{Transparency and Integrity in Research}

The Host Institution must ensure that all necessary authorizations from competent ethical, data protection committees, or other regulatory bodies are obtained for the project. Compliance with the FNR Research Integrity Guidelines and the FNR Ethical Charter is mandatory for all FNR applicants, as outlined on the FNR Homepage. The FNR emphasizes adherence to fundamental ethical principles, including those in the Charter of Fundamental Rights of the European Union, and endorses the European Code of Conduct for Research Integrity and the Singapore Statement on Research Integrity. Any research misconduct, such as non-compliance with ethical regulations, provision of false information, plagiarism, or data falsification, may lead to the rejection of proposals, immediate suspension, or termination of ongoing grants. The FNR reserves the right to impose additional sanctions, such as requesting reimbursement of the grant or taking legal action. Furthermore, all FNR-funded projects must comply with the EU’s General Data Protection Regulation (GDPR) concerning data protection issues. Good practices in data management, protection, and security are expected, in line with the FNR Policy on Research Data Management and the FAIR principles. Beneficiaries and Host Institutions are responsible for ensuring appropriate stewardship and curation of research data generated within FNR-funded projects.

\section{Comment on Resubmission (if applicable)}

\subsection{Evaluation Process and Feedback}

The evaluation process for research proposals involves several key steps to ensure a thorough and fair assessment. Initially, panel experts are required to complete and submit their reviews, providing comments and scoring the proposals at least 10 working days before the panel meeting. During the meeting, these experts present their synthesis of the written evaluations, focusing on the proposal's objectives, strengths and weaknesses, and overall assessment. They also highlight any conflicting statements from reviewers and suggest necessary modifications if applicable.

The panel meeting, organized by the FNR, begins with a ranking of proposals based on the reviews received. Proposals that do not meet the minimal quality criteria are not discussed further, although the justification for their low scoring is verified by the panel. Panel members are reminded to declare any conflicts of interest before proceeding with the discussion of proposals.

The discussion aims to readjust the ratings if necessary, with panel members participating actively. After rating all proposals, the panel deliberates on which should receive funding, considering the potential impact on industry, policy makers, and society. The FNR encourages the dissemination of research to the public and media, emphasizing the importance of generating impact from the project's inception.

Successful proposals form the basis of a grant agreement between the FNR, the Host Institution, and the Applicant, who must agree to the Terms and Conditions. The FNR is committed to the principles of the European Charter for Researchers, ensuring that the research aligns with these standards. The process is designed to support major projects, such as those in collaboration with the Ministry of Foreign and European Affairs and initiatives like the SnT’s Partnership Programme, which addresses key challenges in ICT through collaboration with over 70 organizations.

\subsection{Panel Meeting and Funding Decisions}

The evaluation process for research proposals involves several key stages and considerations. Initially, proposals undergo an administrative eligibility check based on formal requirements. Each eligible proposal is reviewed by two experts or panel members who are closest to the domain of the proposal, although the panel members are generalists and not necessarily experts in the specific domain. The panel, nominated annually by the FNR, discusses each proposal in a meeting, where the panel expert assigned to the proposal presents a synthesis of the written evaluations. This synthesis focuses on the objective of the proposal, fulfillment of selection criteria, and highlights any conflicting statements from reviewers.

During the panel meeting, panel members are encouraged to engage in discussions of all applications, regardless of their field, and to raise any concerns or questions. They are also reminded to declare any conflicts of interest. The panel evaluates the proposals based on strengths and weaknesses, overall assessment, and necessary modifications if applicable. Ethical considerations are addressed if the proposal raises any ethical issues, and the assessment of selection criteria ranges from excellent to fair/poor.

After discussing the evaluation findings, panel members may readjust the rating of the proposal if deemed necessary, providing argumentation for any changes. Once all proposals are rated, the panel discusses which proposals should receive funding. A consensus regarding the funding recommendation requires a majority of two-thirds of panel members, after which the Chair proceeds to a vote. The ‘Panel Conclusion’ is prepared for each project and is the only feedback sent to the applicant. The funding decision is communicated to both the applicant and the supervisor(s).

The FNR, having signed the “Declaration on Research Assessment (DORA),” evaluates the quality and impact of research proposals independently of journal-based metrics, valuing all research outputs. This comprehensive process ensures a fair and thorough assessment of each proposal, ultimately guiding the funding recommendations.

\section{Bibliography (max. 15 references, not included in character limits)}

\subsection{Data Sources and Research Dissemination}

The text collection is built on four heterogeneous data sources: ESA feasibility reports generated during CDF studies, academic publications, books, and Wikipedia pages. These sources are crucial for disseminating research findings and are described in detail in the sections below. The ESA CDF reports, along with Engineering Models, are a primary output of CE studies, summarizing mission objectives, requirements, and design. Academic publications and books are peer-reviewed, ensuring content is verified by humans, while Wikipedia serves as a common open-source knowledge base.

The FNR emphasizes the importance of research outputs impacting science, industry, policy making, and society. To maximize this impact, FNR-funded research results are expected to be disseminated via high-quality, Open Access publications, in line with the FNR Policy on Open Access. The FNR also encourages the deposition of preprints in open access repositories, with costs for project-related publications refundable through the FNR’s Open Access Fund.

The University of Strathclyde KnowledgeBase provides openly accessible resources, and the SnT’s Partnership Programme collaborates with over 70 organizations to address key challenges in ICT, fostering a dynamic interdisciplinary research environment. The Centre has rapidly developed since 2009, launching numerous EU and ESA projects, protecting IP, and creating spin-offs.

Experts during the 2018 survey ranked colleagues as the highest preferred source of information, offering quick answers and tacit knowledge. However, human experts can have biases, making past mission reports and textual documents valuable for heritage information. Internal databases ranked lower than online material due to maintainability issues.

Overall, the integration of diverse data sources and the emphasis on Open Access and collaboration are pivotal in advancing research and its societal impact.

\subsection{Evaluation and Selection Criteria}

The FNR (Fonds National de la Recherche) has implemented a comprehensive approach to evaluating research proposals, emphasizing the quality and impact of research outputs over traditional journal-based metrics. As a signatory of the Declaration on Research Assessment (DORA), the FNR encourages applicants to focus on the scientific content of their work rather than relying on metrics like Journal Impact Factors. This approach values a diverse range of research outputs, including data, software, intellectual property, and the training of young scientists, recognizing their importance across different disciplines and individual contributions.

The selection process for research proposals involves an administrative eligibility check and adheres to international merit review standards. Research integrity is paramount, with strict guidelines in place to prevent misconduct such as plagiarism or data falsification. The FNR also conducts automated plagiarism checks on randomly selected applications to ensure compliance with its Research Integrity Guidelines.

To maximize the impact of research, the FNR requires that results from funded projects be disseminated through high-quality, Open Access publications, in line with its Open Access Policy. The FNR's Open Access Fund can refund costs associated with project-related publications. Additionally, the FNR encourages activities aimed at generating societal impact from the initial project planning stages, ensuring that research benefits science, industry, policy-making, and society at large.

\end{document}
```

<START_OF_AI_MESSAGE>
The draft for your research project essay on Advanced Rocket Engines Development is ready. It has been structured according to your plan and each section has been elaborated upon with detailed content. The document is formatted in LaTeX and adheres to the guidelines provided. Please review the draft and let me know if there are any further adjustments needed. <END_OF_AI_MESSAGE>